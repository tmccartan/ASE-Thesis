% ----------------------------------------------------------------
% Article Class (This is a LaTeX2e document)  ********************
% ----------------------------------------------------------------
\documentclass[12pt]{article}
\usepackage[english]{babel}
\usepackage{amsmath,amsthm}
\usepackage{amsfonts}
% THEOREMS -------------------------------------------------------
\newtheorem{thm}{Theorem}[section]
\newtheorem{cor}[thm]{Corollary}
\newtheorem{lem}[thm]{Lemma}
\newtheorem{prop}[thm]{Proposition}
\theoremstyle{definition}
\newtheorem{defn}[thm]{Definition}
\theoremstyle{remark}
\newtheorem{rem}[thm]{Remark}
\numberwithin{equation}{section}
% ----------------------------------------------------------------
\begin{document}

\title[Forensic Twitter API Anaylsis]{}%
\author{TMcCartan}%
\address{}%
\thanks{}%
\date{}%
% ----------------------------------------------------------------
\begin{abstract}
  Twitter API
  
  Intent
  
  The purpose of this article is to find out the capability of the twitter API and how suitable it will be for acting as the storage and communication method for the thesis. This document will cover the basics of the API such as how one gains access to the API to more advanced aspects like accessing users feeds and posting on behalf of a user. T  
  Gaining access the API
  Image Storage
  REST API
    Requesting a users feed
    Creating a tweet for a user
    Searching/ Retrieving
  Proposed Communication method
  
  
\end{abstract}
\maketitle
% ----------------------------------------------------------------
\section{}



\end{document}
% ----------------------------------------------------------------
